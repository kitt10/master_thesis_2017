\chapter{Code Documentation (API)} \label{app:code_documentation}
The implementation follows the presented methods. The documentation of the main class \texttt{FeedForwardNet()} in the developed \texttt{kitt\_lib} library is given here.

% NeuralNetwork
\noindent\fcolorbox{lightblue}{lightyellow}{
\begin{minipage}{\textwidth}
  \small {\fontfamily{pcr}\selectfont class kitt\_lib.kitt\_net.\textbf{FeedForwardNet()}}
\end{minipage}}
\\
\noindent\fcolorbox{lightblue}{lightgray}{
\footnotesize
\begin{minipage}{\textwidth}
  The main class representing a feedforward neural network.
\begin{itemize}
\item[\textbf{@}] \textbf{hidden} (array-like) : hidden network structure (e.g. [10, 5]); 
\item[\textbf{@}] \textbf{tf\_name} (str) : transfer function name (e.g. 'sigmoid'); 
\end{itemize}
\begin{tcolorbox}[boxsep=0pt,top=10pt,left=10pt,right=10pt, bottom=10pt, arc=0pt, auto outer arc, colback=white, colframe=lightgray]
\begin{itemize}
\item[\textbf{def}] \textbf{fit} : Fits the network to given data and trains the model.
\begin{itemize}
\item[@] X    \hspace{50pt}	: array-like, shape (n\_features, n\_samples)
\item[@] y    \hspace{52pt}	: array-like, shape (n\_classes, 1)
\item[@] val\_x      \hspace{34pt}	: array-like, shape (n\_features, n\_samples)
\item[@] val\_y      \hspace{34pt}	: array-like, shape (n\_classes, 1)
\item[@] learning\_rate      \hspace{2pt}	: learning rate for backpropagation
\item[@] batch\_size      \hspace{14pt}	: mini-batch size for backpropagation
\item[@] n\_epoch      \hspace{22pt}	: number of epochs for backpropagation
\end{itemize}
\item[\textbf{def}] \textbf{predict} : Predicts the probability for each class.
\begin{itemize}
\item[@] x    	\hspace{50pt}	: array-like, shape (n\_features, n\_samples)  
\item[returns] y\_pred\_list \hspace{7pt}	: list, sorted tuples (class, prob) by probability
\end{itemize}
\item[\textbf{def}] \textbf{evaluate} : Returns accuracy and error for given data.
\begin{itemize}
\item[@] x    	\hspace{22pt}	: array-like, shape (n\_features, n\_samples)
\item[@] y    	\hspace{22pt}	: array-like, shape (n\_classes, 1)  
\item[returns] (err, acc) \hspace{23pt}	: MSE' error and accuracy
\end{itemize}
\item[\textbf{def}] \textbf{prune} : Prunes the network.
\begin{itemize}
\item[@] req\_acc    	\hspace{22pt}	: float, required accuracy to be kept
\item[@] n\_epoch    	\hspace{19pt}	: int, number of retraining epochs
\item[@] levels    	\hspace{35pt}	: array-like, pruning levels
\end{itemize}
\item[\textbf{def}] \textbf{copy} : Creates a copy of self.
\begin{itemize} 
\item[returns] net\_copy \hspace{15pt}	: kitt\_net.FeedForwardNet
\end{itemize}
\item[\textbf{def}] \textbf{dump} : Saves the network.
\begin{itemize} 
\item[@] net\_filename \hspace{15pt}	: str, the path to save network as
\end{itemize}
\item[\textbf{def}] \textbf{load} : Loads a network.
\begin{itemize} 
\item[@] net\_filename \hspace{15pt}	: str, the path to network to be loaded
\end{itemize}
\end{itemize}
\normalsize
\end{tcolorbox}
\end{minipage}}